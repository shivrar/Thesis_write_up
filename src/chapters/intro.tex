\let\textcircled=\pgftextcircled
\chapter{Introduction}\label{ch:intro}%    test ~\parencite{pozyx2018pozyx}
%    \lipsum[2-4]
    \initial{I}n recent years Unmanned Aerial Vehicle (UAV) usage has grown exponentially becoming common in industry and households~\cite{custers2016drones}.
    A major part of UAV applications is their ability to localise themselves in the given environment with acceptable precision and accuracy.
    This is a common requirement in any robotic system but UAV's are often limited by strict payload requirements and therefore have to rely on sensors that are both lightweight and robust.
    ~\citep{ardupilotadvanced} gives a good summary of physical components that are used in various vehicles but a UAV system, specifically, a quadrotor system can be summarised as follows:
    \begin{itemize}
        \item Rotor build - This section contains parts that should be researched based on the size and physical requirements of the UAV (drone).
        These include brushless motors, electronic speed controllers and frame size.
        \item Flight Controller Unit (FCU) - This acts as the motherboard and brain of the quadrotor system.
        It collates data from various sensors, sends commands to the motors and, if there is a companion computer attached, it collects and sends data to the FCU.
        Commercial FCU's contain the various control systems and laws required for stable flight and movement.
        Most have an array of sensors built-in.
        \item Sensors - These vary from inertial, positioning, barometric and camera.
        Aside from inertial and barometric sensors that are present in most FCU's, sensors are chosen based on the environment and use of the system.
        \item Companion computer - In some cases, higher-level processing is required by the system to execute autonomy and a secondary computer is used to do this.
        \item Transmitter and Receiver - This is used to implement manual control over the drone by a user.
    \end{itemize}

Further research into sensors, we can classify UAV's based on their operating environment, indoors or outdoors.
These give rise to two forms of localisation and navigation systems:
    \begin{itemize}
        \item Global Positioning Systems (GPS) - As the name suggests this setup uses GPS as well as other sensors.
        \item GPS-denied - These systems do not have access to GPS due to their operating environment.
    \end{itemize}

In outdoor applications, GPS provides a reliable and fairly accurate way to localise with the use of several other sensors.
    However, indoor applications are denied the benefits of GPS and often must use other sensors for the task of localisation.
    Utilizing a similar concept of triangulation used by GPS's a Local Positioning System (LPS) can be used for indoor environments.
    ~\citep{pozyx2018pozyx} has developed a commercial system that utilizes Ultra-WideBand technology (UWB) with a bandwidth of $~\approx 500MHz$.

With indoor environments, users have more control of the environment so an LPS can create a feasible solution for indoor localisation for UAV/robots operating there.
    The bulk of this research was to integrate a commercial LPS directly into an existing FCU to produce accurate position estimates that can be used for autonomy.
    The measurements from the LPS was transformed into observations of the state of the UAV and fused with observations from other sensors.
    This fused pose estimate was then fed into the companion computer for off-board processing.
    \begin{figure}[h!]
        \centering
        \includegraphics[scale=.55]{drone_setup}
        \caption{The typical setup for an autonomous UAV.}
        \label{fig:ds}
    \end{figure}

    Figure:~\ref{fig:ds} shows a typical setup for UAV.
    The parts of the system highlighted in blue represent systems that was worked on during the course of this research.
    The idea is that the system being designed should provide localisation data which should be independent of the rotor build.
    These were further scoped in the upcoming sections but it involved doing a quality exercise of the LPS to determine measurement uncertainty and limitations,
    writing additions or modifying the firmware of the FCU to integrate the LPS and setting up the pipelines for a companion computer to receive the pose estimates and use them.
    Additionally, the overall aim was to get better pose estimates that would form the basis of any autonomous control and navigation.


    % Give intro to LPS, proposed setup and then plan to complete
