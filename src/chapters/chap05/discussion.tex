\chapter{Discussion and Conclusion}\label{ch:discussion}
\initial{I}n Chapters:~\ref{ch:intro} and ~\ref{ch:literature-review} the use and efficacy of using a UWB based system for localisation was explored.
Although many of the research dived into the localisation not many highlighted the use in real-life scenarios and environments.
In the previous research, the systems researched the use in GPS denied environments but not much covered the cases of dynamic obstacles in the environment that provide NLOS and hence errors in TOF calculations.
As mentioned previously, ~\citet{evaluwb} provide a comprehensive analysis and evaluation of localisation based on the Pozyx system but it addressed the issues of static obstacles with NLOS.
The gap of research addressing dynamic obstacles in the environment using the UWB Pozyx sensor is where this research comes in.
The aim of this research can be summarised into the following tasks:
\begin{enumerate}
    \item Evaluate previous work with the Pozyx system and determine a suitable configuration for the test environment.
    \item Show it is possible for an FCU to receive and use the Pozyx sensor readings.
    \item Evaluate an approach to use the sensor readings to get viable position estimates.
    \item Test the discussed approach in non ideal scenarios.
\end{enumerate}

From Section~\ref{sec:technical-design} it is shown that it is possible to pipe the information from the tag to the Pixhawk unit through its I2C interface.
Furthermore, the written test, see Appendix~\ref{subsec:ap_pozyx_test.cpp}, confirms that the Pixhawk's AP\_Beacon subsystem is indeed able to access the Pozyx tag's measurement through the designed interface.
Through the right configuration of user parameters and recompilation of the codebase, it was also confirmed that the new AP\_Beacon library runs successfully with the main code's scheduler to update the beacon's sensor reading periodically so that the other subsystems in the codebase can use.
The core subsystem that uses the beacon's frontend readings would be the Ardupilot's EKF estimation modules.
Researching that subsystem it was confirmed that the beacon frontend was being used to obtain vehicle positions but was not using it in the fusion step of the EKF to estimate position.
At the end of Section~\ref{subsec:sensor-fusion} it was discovered that the EKF required certain checks to be passed before fusing the beacon's  vehicle position and due to the lack of hardware this could not be accomplished.
After reviewing the objectives and scope of this research project, it was determined that the refactor required to get the estimates from the Pixhawk was not necessary.
It was proven that the data can be used on the Pixhawk and the position estimates can be obtained via other methods in order to evaluate the use of the Pozyx system in indoor localisation with persons in the environment.

To evaluate the use of the sensor for indoor localisation multiple scenarios were developed as seen in Chapter~\ref{ch:results-and-analysis}.
Since EKF's are used commonly in navigation and control systems, as seen in the case of Ardupilot, it was utilised in this research in order to get position estimates in the household environment.
The EKF was implemented in two ways as seen in Section~\ref{subsec:sensor-fusion} with one using the tag measurements alone and an embedded one combining dead reckoning data with the tag readings.
From the tag's measurements while a person walked randomly in the environment it was shown that this caused major issues in the readings, particularly when NLOS occurred while the tag was being turned.
This result was expected since the tag was configured to utilise its internal IMU in a filtering algorithm.
The combination of increased TOF readings due to NLOS and the IMU readings whilst in motion gave substantial positioning errors in the range of meters.
These positioning errors can be seen prominently in Figure ~\ref{fig:romi_nlos_1}.
Unreliable readings like these are why fusing multiple measurements of state is necessary.
Dead reckoning is a common way of determining pose estimates in vehicles and robotic systems.
Ardupilot also uses dead reckoning in its EKF so employing one here addresses the capabilities and position estimates that can be possible when the system is finally used on a UAV.
From the distance metrics seen in Table~\ref{tb:results} it is clear to see that a simple fusion of data is able to increase the position estimates greatly with the max distance from line (8.7cm) being better than the quoted accuracy of the Pozyx system ($\pm10$cm) even with NLOS present.
In general, the closer any of the distance metrics are to zero the better the positioning result.

%\chapter{Conclusion}\label{ch:conclusion}
\section{Conclusions}\label{sec:conclusions}
\initial{T}hroughout this research the idea of using a commercially available UWB sensor system for indoor localisation was thoroughly checked.
In addressing the research question and objectives, a software engineering approach was employed to select various hardware options.
Suitable communication interfaces based on previous technical work and existing software solutions was chosen.
In addition to integrating libraries with the existing codebase so it can run seamlessly in the main code and finally run unit and integration tests to ensure everything worked as needed.
Although a viable design was developed and presented it could not be tested due to limitations.
For the evaluation of the Pozyx system another embedded system was used.
From the evaluation a major issue that was immediately identified from the measurements was the large positional error that arose when the system was in motion and there was No Line of Sight between Anchors and the tag.
This showed that the Pozyx system alone cannot be used for localisation and navigation.
However, as with any robotic or UAV system, it is not expected that only one sensor would be used to generate position estimates.
As seen from the Table~\ref{tb:results} the use of an EKF with a simple model and dead reckoning measurements of position greatly improved the position estimates and it is expected that any system that employs this, such as the Ardupilot codebase, should be able to get better position estimates in order to carry out
their autonomous tasks.
Comparing the obtained results with those from ~\citet{di2019evaluation} it can be seen that the max and mean distance errors are similar when operating in the tracking mode in both scenarios.
The results using the EKF with the tag readings from this research achieved an average max and mean error of 0.239m and 0.0743m.
This is an improvement of the results seen in the evaluation by ~\citet{di2019evaluation} which used the Pozyx tag's raw readings.
The results obtained from the combination of the Pozyx tag readings and the dead reckoning data from the mobile robot corroborated with the results from ~\citet{conceiccao2017robot} that used a similar EKF localisation scheme.
In summary, fusing an additional source of information improves the position estimates regardless of the source of the errors.

\subsection{Limitations, Mitigation and Future Work}\label{sec:limitations,-mitigations-and-future-work}
Due to the current pandemic situation, some of the major objectives of the research project had to be changed.
Initially, it was planned that a UAV would be flown autonomously indoors utilising the poses from the FCU in a higher level planner but due to the closing of the laboratories this objective was removed.
There was no drone readily available to test the new software on and there was no way to easily determine the ground truth positions of the UAV while in flight.
Additionally, having no UAV available hindered the testing on the FCU since the FCU had to be in flight in order for the EKF to use the Pozyx readings in the position estimates.
To compensate for the lack of position estimates from the Pixhawk and Ardupilot, another robot was used that provided dead reckoning data.
This was suitable for the scenario since a similar approach is employed in Ardupilot although how the dead reckoning estimates were generated is different due to it being a different vehicle/actuated system.
Furthermore, the lack of an actual laboratory space constrained what tests could be done.
Using a shared space meant that experiments had to be confined to a single counter-top and measurements were made as accurate as possible without the use of any external tools that would be available in the laboratory.
Future work and extension of this research would be to actually test the developed AP\_Beacon system on an UAV and compare the pose estimates while in flight to the ground truth estimates.
Furthermore, the reduced scope objective of actually navigating the UAV would be another extension of this research and work towards a completely viable indoor navigation system.
A final extension of this work would be to extensively test the EKF estimates to track and localise a person within the workspace without being confined to a specific area.
%\lipsum[2-8]
%\lipsum[2-8]