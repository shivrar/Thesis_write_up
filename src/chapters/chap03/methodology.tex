\chapter{Research Methodology}\label{ch:research-methodology}
%\lipsum[2-8]
\initial{F}rom the literature there is a clear lack of results for localisation systems using the commercial Pozyx sensor network with drones in an household environment.
To address the aims and objectives stated in ~\ref{sec:aims_objs}, it is proposed that a lightweight on-board localisation be developed to check the feasibility of minimal hardware solution for an indoor drone.
This would entail connecting a Pozyx tag directly to a FCU and integrate it with the pose localisation system existing already.
To close the loop the pose information should be available in some format that can be used for autonomous control.
Furthermore, the system should be tested in a practical environment, to this end, the Pozyx anchor network was setup in a kitchen.
A kitchen represents one of the highest traffic areas in a household and it contains various materials that will make raw readings from a UWB system noisy and inaccurate.
As a kitchen will contain both dynamic and static obstacles multiple anchor configurations must be tested in order to find optimal anchor positions in this given environment.

\section{Anchor Configurations}\label{sec:anchor-configurations}
Intro to the basic concept, highlight Pietra's paper and how I am using that to phrase and determine the best location
Show pics, diagrams and intial table of results?
