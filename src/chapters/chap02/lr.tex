\chapter{Literature Review}\label{ch:literature-review}
\section{Indoor Localisation Systems}\label{sec:indoor-localisation-sensors}
\subsection*{Passive Systems}
%TODO: PUT IN MORE PAPERS CITING UWB AND WORK WITH UWB/POZYX -> look at the pozyx site
In summary, passive systems do not require the object being tracked to have some of electronics on them to do positioning.
Some examples of passive systems are ~\citep{deak2012survey}:
\begin{itemize}
    \item Computer Vision and Imaging systems.
    \item Tactile and contact sensors.
    \item Attenuation of signals.
    \item Differential air pressure.
\end{itemize}
A common example of computer vision based localisation is to use a setup consisting of multiple cameras in a space trying to detecting a single object.
Using the intrinsic and extrinsic properties of each camera it is possible to determine the transform to the object in a given frame with relatively high accuracy.
A prime example of this is the commercial VICON motion capture systems.
~\cite{aerialrobotsiitk} shows a drone application in which the UAV is positioned via using a VICON motion capture system.
Figure~\ref{fig:vs} shows the simplified setup, it should be noted that the UAV must be equipped with specialised marker that is used to identify it.
Furthermore, the positions are fed to a companion computer connected to the autopilot system.
The VICON setup provides highly accurate positions and is often used to gather ground truth positional data to compare to other positioning systems.
The additional requirements of the VICON systems, however, do not make it feasible for indoor applications.

\begin{figure}[h!]
    \centering
    \includegraphics[scale=.45]{vicon_setup}
    \caption{VISON setup for position of a UAV. (\url{https://aerial-robotics-iitk.gitbook.io/wiki/estimation/setup-with-vicon})}
    \label{fig:vs}
\end{figure}

\subsection*{Active Systems}
In contrast to passive systems, active systems have the object being positioned equipped with electronics.
Many indoor localisation techniques use this and some examples are ~\cite{deak2012survey}
\begin{itemize}
    \item Radio-frequency identification
    \item UWB
    \item Wireless Local Area Network
    \item Bluetooth Low energy (BLE)
\end{itemize}
Many of these setups use an anchor and tag configuration.
The tag receives signals from multiple anchors and triangulates the tag.

An approach using and comparing UWB and BLE is developed by ~\cite{findobjs} to do localisation in a museum.
Both methods are combined with a dead reckoning system to improve accuracy.
Six paintings are equipped with both a BLE and UWB tag.
The test setup first did a calibration where both sensors were placed at fixed points in the museum with a clear line of sight to each tag.
From initial ranging performance, the UWB setup was shown to perform better with a distance variance of $\pm0.4m$ while the BLE setup had errors of over $10m$.
It is noted by the authors that a ranging approach with BLE is challenging since it uses an RSSI method but can match the accuracy of the UWB setup when combined a dead reckoning system after some initial steps by the user.

\begin{figure}[h!]
    \centering
    \includegraphics[scale=.45]{uwb_vs_ble}
    \caption{Setup used to compare UWB and BLE performance in a museum.}
    \label{fig:uwbvsble}
\end{figure}

\section{Pozyx - Behind the Scenes}\label{sec:pozyx---behind-the-scenes}
For this research project a commercially available, active UWB sensor system is proposed to aid in localisation of an indoor UAV system.
\begin{figure}[h!]
    \centering
    \includegraphics[scale=.7]{lr/uwb_block}
    \caption{Simplified Block Diagram of the Pozyx tag.}
%    \label{fig:twr}
\end{figure}
\subsection*{Ultra-WideBand (UWB)}
The core of the Pozyx system operates using an UWB approach.
UWB is a short range, low energy, high bandwidth communication radio technology.
Radio waves travel at the speed of light ($c=299792458ms^{-1}$) so using a TOF approach the range between a tag and an anchor can be obtained simply by:
\[
    d = c*TOF
\]

Knowing the position of each anchor in a given reference frame, ~\citet{evaluwb} discuss a method to use raw range readings in order to determine the position of the tag.
The positions can be described by the following system of equations:
        \begin{equation}
                \left[
            \begin{array}{c}
                (x - x_1)^2 + (y - y_1)^2 + (z - z_1)^2 = d_1^2\\
                \vdots\\
                (x - x_n)^2 + (y - y_n)^2 + (z - z_n)^2 = d_n^2
            \end{array}
                \right]
        \end{equation}
        Where $n$ represents an index of an anchor and $(x,y,z)$ represents the position of the tag.
        This can be converted into matrix form of \textbf{A.x = B}:
        \begin{equation}
            \left[
            \begin{array}{c}
                1 - 2x_1 -2y_1 -2z_1\\
                \vdots\\
                1 - 2x_n -2y_n -2z_n
            \end{array}
            \right]
        *
            \left[
                \begin{array}{c}
                    x^2 + y^2 +z^2\\
                    x\\
                    y\\
                    z\\
                \end{array}
            \right]
            =
        \left[
            \begin{array}{c}
                d_1^2 - x_1^2 - y_1^2 -z_1^2\\
                \vdots\\
                d_n^2 - x_n^2 - y_n^2 -z_n^2
            \end{array}
        \right]
       \end{equation}

        The position of the tag can be calculated as:
        \[
            \hat{x} = (A^{T}A)^{-1}A^{T}B
        \]

From the algorithms and work presented we can confirm that the Pozyx system can achieve an accuracy of $\pm10cm$ in standard environments with LOS.
The work confirms that LOS of the anchors to tag is a major factor in accuracy and this will be taken into account when anchors are being placed in the research.
Furthermore, the work mentioned in this section proposes an algorithm with raw range readings in order to do localisation, my research will focus on the integration of the Pozyx tag with a standard FCU, so the pose data coming directly from the tag can be used.
The pose can be obtained via two modes: 1). A pure Two Way Ranging (TWR) Approach or 2.) A tracking approach using a Kalman prediction filter in addition to the TWR pose.
%TODO: MARK HERE WITH MORE PAPERS EXPANDING UWB
\subsection*{Two Way Ranging (TWR)} % TODO: Look at pulling this put into an appendix?
TWR is a ranging method that utilises TOF and delays during transmission of a packet in order to determine the range between a tag and anchor.
Figure~\ref{fig:twr} provides a simple illustration of how this works.
The distance for an individual tag and an anchor can be obtained by:
\[
    d=c.\frac{(TT2-TT1)-(TA2-TA2)}{2}
\]
This is repeated for each anchor and then the position of the tag can be determined via trilateration.
Geometrically, the position of the tag can be described as the point intersection of all the circles with distance,d , from the tag.
This can be seen in Figure~\ref{fig:trilat}.
\begin{figure}[h!]
    \centering
    \includegraphics[scale=.7]{lr/TWR}
    \caption{Packet transfer in TWR.}
    \label{fig:twr}
\end{figure}

\begin{figure}[h!]
    \centering
    \includegraphics[scale=.7]{lr/trilat}
    \caption{4 Anchor and 1 Tag trilateration in 2D.}
    \label{fig:trilat}
\end{figure}


\subsection*{Sensor Fusion}
From the previous section we have described the basic principle of operation of the Pozyx system.
With a tag we are able to at least determine a rough estimate of the position of a tag in a given reference frame.
Additionally, the tag has an Inertial Measurement Unit (IMU), consisting of an Accelerometer, Gyroscope and Magnetometer, and an Altimeter.
These are used in one of the operational modes of the Pozyx system in order to improve accuracy.
The idea of sensor fusion is to combine multiple sources of data in order to get a fairly accurate estimate of the pose of the system,
The measurements from the tag can be combined with the sensors onward a FCU in order to achieve this.
The de-facto sensor fusion technique is called the Extended Kalman Filter (EKF).
~\citet{simpleekf} present a useful description and example of how the EKF works.
Algorithmically, the EKF is a recursive process using predictions based on the dynamics of the vehicles and updating the estimate based on these predictions and measurements from various sources.
The major requirement for the EKF is that the process model and the measurement model are differentiable.
The steps for the EKF are as follows:
\begin{enumerate}
    \item Provide and initial estimate for the state, $\hat{x}^+_k$, and the prediction error, $P^+_k$.
    \item Compute the Kalman gain, $K_k = P^+_{k}H_k^T(H_k P^+_{k}H_k^T + R)^{-1}$
    \item Update the estimate with measurement $z_k$, $\hat{x}_k=\hat{x}^+_k + K_k(z_k - h(\hat{x}^+_k))$
    \item Update the prediction error, $P_k = (I - K_k H_k)P^+_{k}$
    \item Project the state ahead, $\hat{x}^+_{k+1} = f(\hat{x}_k, u_k, w)$
    \item Project the Prediction error ahead, $P^+_{k+1} = A_k P_k A_k^T + Q_k$
    \item Repeat from step 2.
\end{enumerate}
Where K is the Kalman gain, R is the Measurement Noise Covariance Matrix, Q is the Process Noise Covariance matrix.
~\citet{conceiccao2017robot} present a method for using the Pozyx in an outdoor environment, a similar methodology would be adapted for this work with focus on indoor limitations.
%TODO:   ^ This paper can be expanded on a little more.

\subsection*{Pozyx - Arduino Implementation}
Before continuing it should be noted that ~\citet{ardupilotarduino} has addressed the idea of combining the Pozyx system with a Flight controller using the Ardupliot firmware.
The implementation uses the Pozyx tag's compatibility with the Arduino UNO R3 or R2 pin layout.
The Pozyx Arduino library is used to gather the relevant information from the Pozyx tag and then send it via serial to the FCU.
This research aims to bridge this gap in the hardware and remove the need for the Arduino UNO for the indoor navigation.
The major driving force of this is to minimize the amount of extra hardware that should be mounted on and indoor UAV.
% TODO: Additional closing paragraph to help solidfy my options for the methodlogy and guide the reader. -> Why am i doing the technical work I am doing.
% TODO: Major gaps -> Major gap in feasible and usable results for indoor loclisation of drones in a real scenario -> Kitchen for eg.
% TODO: Find papers that show these lacking stuff.